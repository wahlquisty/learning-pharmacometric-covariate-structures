\documentclass[tikz, margin=2mm] {standalone}
\usetikzlibrary{arrows.meta, % for edge arrows 
               matrix,      % nodes will set as matrix elements
               positioning,
               calc,
               }
\usetikzlibrary{narrow}
\usepackage{tikz-qtree}

% Handle the yellow background
\pgfdeclarelayer{background}
\pgfdeclarelayer{foreground}
\pgfsetlayers{background,main,foreground}


\begin{document}
    \begin{tikzpicture}[
           >=narrow, semithick,
              plain/.style = {draw=none, fill=none, yshift=10mm,
                text width=9ex,  align=center},% for text in images, 
              ec/.style = {draw=none},% for emty cells
               net/.style = {% for matrix style
         matrix of nodes,
          nodes={circle, draw, semithick, minimum size=5mm, inner sep=0mm},% circles in image
       nodes in empty cells,% for not used cells in matrix
       column sep = 4.5mm, % distance between columns in matrix 
        row sep = -2mm  % distance between rows in matrix
        },
]
        \matrix[net] (m)% m is matrix name, it is used for names of cell: firs has name m-1-1
                        % in empty space between ampersands will show circles: 
                        % i.e.: nodes of the neural network
        {
        |[plain]|  &[-4mm] |[plain]|  &[-8mm] |[plain]| &[-4mm] |[plain]|  &[-8mm] |[plain]|  &[-5mm] |[plain]|   &[-8mm] |[plain]| \\ %
               &        &        &         &        & |[ec]| & |[ec]| \\
        |[ec]| & |[ec]| & |[ec]| & |[ec]|  & |[ec]| & |[ec]| & |[ec]| \\
               &        & |[ec]| &         & |[ec]| & |[ec]| & |[ec]| \\
        |[ec]| & |[ec]| &        & |[ec]|  &        & |[ec]| & |[ec]| \\
               &        & |[ec]| &         & |[ec]| &        &        \\
        |[ec]| & |[ec]| & |[ec]| & |[ec]|  & |[ec]| & |[ec]| & |[ec]| \\
               &        & |[ec]| &         & |[ec]| & |[ec]| & |[ec]| \\
        |[ec]| & |[ec]| &        & |[ec]|  &        & |[ec]| & |[ec]| \\
               &        & |[ec]| &         & |[ec]| & |[ec]| & |[ec]| \\
        |[ec]| & |[ec]| & |[ec]| & |[ec]|  & |[ec]| & |[ec]| & |[ec]| \\
        %|[ec]| & |[ec]| &[-8mm] |[ec]| & |[ec]|  &[-8mm] |[ec]| & |[ec]| \\
        %|[ec]| & |[ec]| &[-8mm] |[ec]| & |[ec]|  &[-8mm] |[ec]| & |[ec]| \\
        };
        
        % inputs
        % \foreach \in [count=\ir from 1] in {3,5,7,9}
        % \draw[black,thick,<-] (m-\in-1.west) -- node[above] {\scriptsize{Input \ir}} +(-10mm,0);
        \draw[black,<-] (m-2-1.west) -- node[above] {\scriptsize{$\varphi_1$=\;age}} +(-13mm,0);
        \draw[black,<-] (m-4-1.west) -- node[above] {\scriptsize{$\varphi_2$=\;weight}} +(-13mm,0);
        \draw[black,<-] (m-6-1.west) -- node[above] {\scriptsize{$\varphi_3$=\;BMI}} +(-13mm,0);
        \draw[black,<-] (m-8-1.west) -- node[above] {\scriptsize{$\varphi_4$=\;gender}} +(-13mm,0);
        \draw[black,<-] (m-10-1.west) -- node[above] {\scriptsize{$\varphi_5$=\;AV}} +(-13mm,0);

% connections between nodes in the first and second layer
\foreach \j in {2,4,6,8,10} % nodes from
{
\foreach \k in {2,4,6,8,10} \draw[black,-] (m-\j-1) -- (m-\k-2); % nodes to
}

% connections between nodes in the second and third layer (function layer)
\draw[black,-] (m-2-2) -- (m-2-3); % unit
\node[] at (m-2-3) {\scriptsize{$1$}};
\draw[black,-] (m-4-2) -- (m-5-3); % mult
\node[text height=2mm] at (m-5-3) {\scriptsize{$\times$}};
\draw[black,-] (m-6-2) --  (m-5-3); % mult
\draw[black,-] (m-8-2) --  (m-9-3) node[above, xshift=-4mm, yshift=1mm]{\scriptsize{$a$}}; % pwr
\draw[black,-] (m-10-2) --  (m-9-3) node[below, xshift=-4mm, yshift=-1mm]{\scriptsize{$b$}}; % pwr
\node[] at (m-9-3) {\scriptsize{$a^b$}};

% connections between nodes in the third and fourth layer
\foreach \j in {2,5,9}
{
\foreach \k in {2,4,6,8,10} \draw[black,-] (m-\j-3) -- (m-\k-4);
}

% connections between nodes in the fourth and fifth layer (function layer)
\draw[black,-] (m-2-4) -- (m-2-5); % unit
\node[] at (m-2-5) {\scriptsize{$1$}};
\draw[black,-] (m-4-4) -- (m-5-5); % mult
\node[text height=2mm] at (m-5-5) {\scriptsize{$\times$}};
\draw[black,-] (m-6-4) --  (m-5-5); % mult
\draw[black,-] (m-8-4) --  (m-9-5) node[above, xshift=-4mm, yshift=1mm]{\scriptsize{$c$}}; % div
\draw[black,-] (m-10-4) -- (m-9-5) node[below, xshift=-4mm, yshift=-1mm]{\scriptsize{$d$}}; % div
\node[] at (m-9-5) {\scriptsize{$\frac{c}{d+1}$}};

% connections between nodes in the fifth and last layer
\foreach \j in {2,5,9}
{
\foreach \k in {6} \draw[black,-] (m-\j-5) -- (m-\k-6);
}

% Last layer
\draw[black,-] (m-6-6) -- (m-6-7) node[above, xshift=-5mm, yshift=0mm]{\scriptsize{$e$}};
\node[] at (m-6-7) {\scriptsize{$|e|$}};

% output
\draw[black,->] (m-6-7.east) -- node[above]  {\scriptsize{$\theta_2=k_{12}$}} +(10mm,0);

% Compute the coordinates for the background and the text above 
% the background

\coordinate(b1) at ($(m-10-5) + (9mm,-5mm)$);
\coordinate(b2) at ($(m-1-4) + (-6mm,6mm)$);

\coordinate(b3) at ($(m-10-3) + (9mm,-5mm)$);
\coordinate(b4) at ($(m-1-2) + (-6mm,6mm)$);

\end{tikzpicture}

\end{document}